% SFD pour le projet VirtualWars du S2

% Package utilisé --------------------------
\documentclass[12pt,a4paper]{article}
\usepackage{color}
\usepackage[T1]{fontenc}
\usepackage{lmodern}
\usepackage[utf8]{inputenc}
\usepackage[francais]{babel}
\usepackage{listings}
\usepackage{amssymb}
\usepackage{xspace}
\usepackage{graphicx}
\usepackage{hyperref}
\usepackage{makeidx}
\usepackage{tikz}
\usepackage{caption}
\usepackage{float}

\lstset{
	language = Java,
	numbers=left,
	numbersep=7pt,
	backgroundcolor=\color{white},
	frame=single,
}

\definecolor{pblue}{rgb}{0.13,0.13,1}
\definecolor{pgreen}{rgb}{0,0.5,0}
\definecolor{pred}{rgb}{0.9,0,0}
\definecolor{pgrey}{rgb}{0.46,0.45,0.48}

\lstset{language=Java,
  showspaces=false,
  showtabs=false,
  breaklines=true,
  showstringspaces=false,
  breakatwhitespace=true,
  commentstyle=\color{pgreen},
  keywordstyle=\color{pblue},
  stringstyle=\color{pred},
  basicstyle=\ttfamily,
  moredelim=[il][\textcolor{pgrey}]{$$},
  moredelim=[is][\textcolor{pgrey}]{\%\%}{\%\%}
}
%-------------------------------------------

%Page titre --------------------------------
\title{VirtualWars \\Spécification fonctionelle détaillé}
%\subtitle{Spécification Fonctionnelle Détaillée} % Apparament subtitle n'existe pas
\author{Nicolas Mauger \thanks{Merci aux trois autres cavaliers (et le majordome)}
\\DUT informatique de Lille, semestre 2}
\date{\today \\V2.2}
%--------------------------------------------

\begin{document}
%Page titre --------
\maketitle \newpage
%-------------------

%Sommaire ----------
\tableofcontents \newpage
%-------------------

\begin{abstract}
Dans le cadre du deuxième semestre, nous réalisons un jeu de stratégie dans lequel des robots s'affrontent sur un plateau.
\end{abstract}

\begin{quotation}
Ce SFD est rédigé en \LaTeX\ avec l'aide des recommandations 830 et 1233a de l'IEEE. Pour plus d'information \href{http://www.math.uaa.alaska.edu/~afkjm/cs401/IEEE83.pdf}{lien vers la documentation  officiel}.
\end{quotation}

\part{Introduction}
		\section{Objectif}
			Pour éviter les morts inutiles, l'ONU a décidé de créer une application informatique pour remplacer les affrontements réels. L'idée est de développer un environnement virtuel qui permet à deux pays de s'affronter par l'intermédiaire de robots sans engager de troupes sur le terrain.
		\section{Définitions, acronymes et abréviations}
			\noindent
			Tireur: robot léger, capable de tirer à une courte distance. \\
			Piégeur: robot léger, capable de poser des mines. \\
			Char: robot lourd, capable de tirer à une distance plus grande que le tireur. \\
			I.H.M: Interactions homme-machine. Moyens et outils pour contrôler le programme.	\\
			pts : points d'énergie du robot.
			\indent
		\section{Présentation du système}
			Les pays s'affrontent en faisant combattre des robots sur un plateau. Le jeu se déroule en tour par tour. Au départ les robots se trouvent dans leurs bases respectives. A chaque tour de jeu, chaques équipes choisissent un de ses robots pour réaliser une action (déplacement ou attaque). Au cours de la partie chaques équipes doivent conserver au moins un robot hors de sa base. Il existe 3 catégories de robots: tireur, piégeur et char. Et chaque joueur choisit quels robots constituent son équipe, cela fait partie de sa stratégie pour remporter la victoire. La partie se termine dès qu'une des deux équipes ne possède plus de robot vivant.

\newpage				
\part{Description générale}
	Dans cette première étape, l'équipe est constituée d'un robot de chaque type, au fur et à mesure de leur définition. Le contrôle des robots est purement manuel (menu texte), les robots seront dotés d'un "cerveau" mettant en œuvre une tratégie au jalon 2. \\
	Au jalon 2 le jeu sera en mode graphique, et on contôlera leurs actions et déplacement grâce à la souris.
	\section{Règles du jeu}
		\subsection{Quelques règles générales importantes}
			Avant de lancer le jeu, on pendra soin de vérifier si il existe un chemin entre les deux bases malgré les obstacles non déstructibles. Quand le jeu commence, tout les robots sont dans la base. Il est présisé dans le cahier des chages que les joueurs doivent en permance un robot hors de sa base. Il faut donc forcer le joueur à sortir un robot de sa base au premier tour, et ne pas pouvoir renter ce robot si tout les autres sont déjà dans la base. \\
			Quelques règles d'affichage: Le joueur ne voit que ces propres mines. Lorsque un robot rentre dans sa base, on ne voit plus que la case et plus le robot.
		\subsection{Règles de déplacement des robots}
			Les chars ne se déplacent qu'en ligne droite, horizontalement ou verticalement de deux cases:
			\begin{itemize}
				\item Un char ne peut pas délibérémet se déplacer que d'une seule case, mais il peut y être contraint par les limites du plateau ou un obstacle.
				\item Un char ne peut passer que par des cases vides ou mines. Notons que ce n'est pas un cavalier: il ne peut pas sauter au dessus des ennemis ou des obstacles. Si il recontre une mine sur sont passage, sn déplacement est pris en compte et il arrive quand même à destination. Il subit évidemment les dégâts de la mine. \\
			\end{itemize}
				Les tireurs et piégeurs ont les mêmes règles de déplacement. Il se déplacent que d'une seul case dans les 8 cases autour d'eux.
			\begin{itemize}
				\item Notons que lorqu'un robot se déplace d'une case en diagonale, il ne fait pas le trajet d'un case un avant + une case de côté. Il peut s'échapper d'une situation ou il entouré de 4 mines sans subir de dégât.
			\end{itemize}

		\subsection{Règles d'attaque des robots}
			Le tireur ne peux tirer qu'en ligne droite et a une portée de 3 cases, il pert 2 pts en tirant.\\
			Le char ne peux tirer qu'en ligne droite et a une portée de 10 cases, il pert 1 pts en tirant. \\
			Pour le tireur et le char:
			\begin{itemize}
				\item Le tir s'arrete sur le premier ennemie rencontré.
				\item Le tir n'est possible qu'en direction d'une cible identifié. C'est à dire que l'on peut tirer uniquement sur un robot ennemie. Le tir sur des robots alliés n'est pas autorisé.
			\end{itemize}
			Le piègeur pose des mines sur les 8 cases autour de lui. Il pert 2 pts si il pose ou se déplace sur une mine.
			\begin{itemize}
				\item Une seul mine peut-être posée par tour.
				\item Le piègeur dispose de 10 mines au départ, il doit retouner à la base pour reconstituer sur stock. Une fois à la base, il regagne toutes ces mines instantanément.
				\item Un piègeur qui passe sur une mine a le même comportement que les autres robots.
			\end{itemize}
		\subsection{Illustations des capacités de chaques robots}
			%illustration du tireur
			\begin{figure}[!h]
				\includegraphics[scale=1]{T.png}
				\caption{Exemple d'utilisation du Tireur}
				\label{Exemple d'utilisation du Tireur}
			\end{figure}

			 %illustration du Char
			\begin{figure}[!h]
				\includegraphics[scale=1]{C.png}
				\caption{Exemple d'utilisation du Char}
				\label{Exemple d'utilisation du Char}
			\end{figure}
	
			%illustration du Piegeur
			\begin{figure}[!h]
				\includegraphics[scale=1]{P.png}	
				\caption{Exemple d'utilisation du Piègeur}
				\label{Exemple d'utilisation du Piègeur}
			\end{figure}
				
		\subsection{Récapitulatif des valeurs associées aux actions}
		\hspace{-3cm}
			\begin{tabular}{|p{1.8cm}||p{1cm}|p{2cm}|p{1.5cm}|p{2cm}|p{1.5cm}|p{1.5cm}|p{1.5cm}|p{2cm}|}
					\hline
			 	   & Portée (en case) & Déplacement (en case) & Energie Initial (en pts) & Base (en gain de pts/tour) & Miner (coût en pts) & Tirer (coût en pts) & Avancer ( en pts) & Dégâts du tir ou de la mine (en pts) \\
			 	  \hline \hline
			 	  Tireur 	& 3 	& 1 & 40 & +2 & NA & -2 & -1 & -3 \\
			 	  \hline
			 	  Piégeur & 1 	& 1 & 50 & +2 & -2 & NA & -2 & -2 \\
			 	  \hline
				  Char 		& 10 	& 2 & 60 & +2 & NA & -1 & -5 & -6 \\
					\hline
			\end{tabular}

		\subsection{Règles de comportement des bases}
			Les bases permettent aux robots de regagner de l'énergie quand cette-ci est basse. Voici quelques règles qu'elles doivent respecter:
			\begin{itemize}
				\item Dans leurs bases les robots ne peuvent pas être touchés.
				\item Dans leurs bases les robots sont invisibles
				\item Les bases ne peuvent pas donner plus de pts qu'ont les robots initialement.
				\item Les robots ennemies ne peuvent pas enter dans la base.
				\item Au moins un robot par équipe doit être en dehors de la base à chaque instant.
	\subsection{Règles de comportement des obstacles}

	\section{Modes et états}
		Dans la version texte l'écran d'accueil, le paramétrage de la partie, son déroulement, les messages d'erreur, et l'écran de fin de partie sont affichés en texte dans la sortie standard (le terminal de l'utilisateur).
		\subsection{L'écran d'accueil}
			\subsubsection{Mode texte}
					%tableau--------
					\hspace{-3cm}
					\begin{tabular}{|p{1cm}|p{2.5cm}|p{1.5cm}|p{3cm}|p{2cm}|p{3cm}|p{3cm}|} %16cm
						\hline
							Label & Libellé & Type & Taille maximum & Est critique & Valeur par défault & Commentaire \\
						\hline \hline
							M1 & Message d'accueil & Text & une ligne & oui & Bienvenue dans Virtual Wars! & C'est le message d'accueil du jeu \\
							\hline
							M2 & Question règles & Text & NA & oui & Voulez-vous consulter les règles? (o-n)& On rappelle ici les règles du jeu pour les débutants \\
							\hline
					\end{tabular}
					\captionof{table}{Information présente sur l'écran d'accueil texte}
					\label{Information présente sur l'écran d'accueil texte}
					%---------------
				\subsubsection*{Maquette}
					\begin{lstlisting}
Bienvenue dans Virtual Wars!
Voulez-vous consulter les regles? (o-n)
					\end{lstlisting}

			\subsubsection{Mode graphique}
				\hspace{-3cm}
					\begin{tabular}{|p{1cm}|p{2.5cm}|p{1.5cm}|p{3cm}|p{2.5cm}|p{2.5cm}|p{3cm}|} %16cm
						\hline
							Label & Libellé & Type & Taille maximum & Est critique & Valeur par défault & Commentaire \\
						\hline \hline
							B1 & Jouer & bouton & NA & oui & "Jouer" & C'est le bouton pour faire une partie simple \\
							\hline
							B2 & Charger & bouton & NA & oui & "Charger" & C'est le bouton pour charger un fichier de sauvegarde \\
							\hline
							B3 & Multijouer & bouton & NA & non & "Multijoueur" & C'est le bouton pour lancer une partie en réseau \\
							\hline
					\end{tabular}
					\captionof{table}{Information présente sur l'écran d'accueil graphique}
					\label{Information présente sur l'écran d'accueil graphique}
					%---------------
				\newpage
				\subsubsection*{Maquette}
					\begin{figure}[!h]
						\includegraphics[scale=1]{EA.png}
						\caption{Maquette de l'écran d'accueil graphique}
						\label{Maquette de l'écran d'accueil graphique}
					\end{figure}


		\subsection{L'écran de paramétrage}
			\subsubsection{Mode texte}
				%tableau--------
				\hspace{-3cm}
				\begin{tabular}{|p{1cm}|p{2.5cm}|p{1.5cm}|p{3cm}|p{2.5cm}|p{2.5cm}|p{3cm}|} %16cm
				\hline
						Label & Libellé & Type & Taille maximum & Est critique & Valeur par défault & Commentaire \\
					\hline \hline
						M3 & Hauteur du plateau & Text & une ligne & oui & 15 & le plateau est obligatoirement rectangulaire \\
						\hline
						M4 & Largeur du plateau & Text & une ligne & oui & 10 & 15 et 10 sont en valeur par défaut car données en exemple \\
						\hline
						M5 & Nombre d'obsacle & Text & une ligne & oui & 7 & 7 est la valeur par défaut car donnée en exemple \\
						\hline
						M6 & Validation & Text & une ligne & oui & Le paramétrage  de la partie est fini. Appuyer sur une touche pour commencer & Confirmation des paramètres choisis et lancement de la partie\\
					\hline
				\end{tabular}
				\captionof{table}{Informations présentes sur l'écran de paramétrage texte}
				\label{Informations présentes sur l'écran de paramétrage texte}
				%---------------				
				\subsubsection*{Maquette}
					\begin{lstlisting}
Hauteur du plateau? (5-50)
Largeur du plateau? (5-50)
Nombre d obstacle en \% ? (0-50)
Vous avez choisi un plateau avec une dimention de X sur Y avec n\% obstacles. Appuyer sur une touche pour commencer.
					\end{lstlisting}

			\subsubsection{Mode graphique}
				%tableau--------
				\hspace{-3cm}
				\begin{tabular}{|p{1cm}|p{2.5cm}|p{2.5cm}|p{2cm}|p{2.5cm}|p{3cm}|p{3cm}|} %16cm
				\hline
						Label & Libellé & Type & Taille maximum & Est critique & Valeur par défault & Commentaire \\
					\hline \hline
						P1 & Hauteur du plateau & Potentiomètre & NA & oui & 15 & le plateau est obligatoirement rectangulaire \\
						\hline
						P2 & Largeur du plateau & Potentiomètre & NA & oui & 10 & 15 et 10 sont en valeur par défaut car données en exemple \\
						\hline
						P3 & Nombre d'obsacle & Potentiomètre & NA & oui & 7 & 7 est la valeur par défaut car donnée en exemple \\
						\hline
						V1 & Volume de la musique & Potentiomètre & NA & non & 50\% du volume max & NA\\
						\hline
						V2 & Volume des effets & Potentiomètre & NA & non & 50\% du volume max & NA\\
						\hline
						B4 & Validation & bouton & NA & oui & NA & Apres avoir cliqué sur ce bouton, la partie commence\\
					\hline
				\end{tabular}
				\captionof{table}{Informations présentes sur l'écran de paramétrage graphique}
				\label{Informations présentes sur l'écran de paramétrage graphique}
				%---------------
				\newpage
				\subsubsection*{Maquette}
					\begin{figure}[!h] %on ouvre l'environnement figure
						\includegraphics[scale=1]{EP.png}
						\caption{Maquette de l'écran de paramétage graphique}
						\label{Maquette de l'écran de paramétage graphique}
					\end{figure}
					
			
		\subsection{L'écran de jeu}
			\subsubsection{Mode texte}
			%tableau--------
				\hspace{-3cm}
				\begin{tabular}{|p{1cm}|p{2.5cm}|p{1.5cm}|p{2.5cm}|p{2.5cm}|p{3cm}|p{3cm}|} %16cm
					\hline
						Label & Libellé & Type & Taille maximum & Est critique & Valeur par défault & Commentaire \\
					\hline \hline
						A & Plateau de jeu & Text & Selon les valeurs choisi par l'utilisateur & oui & NA & Le plateau sera représenté en ASCII art \\
						\hline
						S1 & Sélection du robot & Text & une ligne & oui & Quel robot voulez-vous jouer? (t,p ou c) & NA \\
						\hline
						S2 &Sélection de l'action & Text & une ligne & oui & Voulez-vous tirer ou vous déplacer? (t ou d) &  NA \\
						\hline
						S3 & Sélection de la direction & Text & une ligne & oui & Dans quelles directions voulez-vous aller? (1-8) & NA\\
					\hline
				\end{tabular}
				\captionof{table}{Informations présentes sur l'écran de jeu texte}
				\label{Informations présentes sur l'écran de jeu texte}
				%---------------
					\newpage
					\subsubsection*{Maquette}
					On a ici une maquette le l'affichage du plateau de jeu 6x6 en mode text. Une des équipes a ces pièces en minuscule, l'autre équipe en majuscule. \\
					Légende: B base, T tireur, P piègeur, M mine, C char, \# obstacle
		\begin{lstlisting}
     1   2   3   4   5   6
   +---+---+---+---+---+---+
 A | B |   |   |   | # |   |
   +---+---+---+---+---+---+
 B |   |   | T |   |   |   |
   +---+---+---+---+---+---+
 C |   |   |   |   |   | # |
   +---+---+---+---+---+---+
 D |   | P | # |   |   | t |
   +---+---+---+---+---+---+
 E | M | # |   |   |   |   |
   +---+---+---+---+---+---+
 F |   | c |   |   |   | b |
   +---+---+---+---+---+---+
		\end{lstlisting}

			\subsubsection{Mode graphique}
				%tableau--------
				\hspace{-3cm}
				\begin{tabular}{|p{1cm}|p{2.5cm}|p{1.5cm}|p{2.5cm}|p{2.5cm}|p{3cm}|p{3cm}|} %16cm
					\hline
						Label & Libellé & Type & Taille maximum & Est critique & Valeur par défault & Commentaire \\
					\hline \hline
						A & Plateau de jeu & Text & Selon les valeurs choisi par l'utilisateur & oui & NA & Le plateau sera représenté en ASCII art \\
						\hline
						S1 & Sélection du robot & Text & une ligne & oui & Quel robot voulez-vous jouer? (t,p ou c) & NA \\
						\hline
						S2 &Sélection de l'action & Text & une ligne & oui & Voulez-vous tirer ou vous déplacer? (t ou d) &  NA \\
						\hline
						S3 & Sélection de la direction & Text & une ligne & oui & Dans quelles directions voulez-vous aller? (1-8) & NA\\
					\hline
				\end{tabular}
				\captionof{table}{Informations présentes sur l'écran de jeu texte}
				\label{Informations présentes sur l'écran de jeu texte}
				%---------------
				\subsubsection*{Maquette}
					\begin{figure}[!h] %on ouvre l'environnement figure
						\includegraphics[scale=1]{EJ1.png}
						\caption{Maquette du plateau de jeu graphique}
						\label{Maquette du plateau de jeu graphique}
					\end{figure}
					\newpage
					\begin{figure}[!h] %on ouvre l'environnement figure
						\includegraphics[scale=1]{EJ2.png}
						\caption{Maquette du plateau de jeu + sélection graphique}
						\label{Maquette du plateau de jeu + sélection graphique}
					\end{figure}
					\begin{figure}[!h] %on ouvre l'environnement figure
						\includegraphics[scale=1]{EJ3.png}
						\caption{Maquette du plateau de jeu + menu déroulant}
						\label{Maquette du plateau de jeu + menu déroulant}
					\end{figure}
					\begin{figure}[!h] %on ouvre l'environnement figure
						\includegraphics[scale=1]{EJ4.png}
						\caption{Maquette du plateau de jeu + exemple message d'erreur}
						\label{Maquette du plateau de jeu + exemple message d'erreur}
					\end{figure}

		\subsection{L'écran de fin de partie}
			\subsubsection{Mode texte}
				%tableau--------
					\hspace{-3cm}
					\begin{tabular}{|p{1cm}|p{2.5cm}|p{1.5cm}|p{2.5cm}|p{2.5cm}|p{3cm}|p{3cm}|} %16cm
						\hline
						Label & Libellé & Type & Taille maximum & Est critique & Valeur par défault & Commentaire \\
						\hline \hline
						M7 & Message de fin & Text & une ligne & oui & Game Over & C'est le message de fin \\
						\hline
						M8 & Rejouer & Text & une ligne & oui & Voulez-vous refaire une partie? (o-n)& Permet à l'utilisateur de refaire une partie sans relancer le programme. \\
						\hline
					\end{tabular}
					\captionof{table}{Informations présentes sur l'écran d'accueil texte}
					\label{Informations présentes sur l'écran d'accueil texte}
					%---------------
					\subsubsection*{Maquette}
					\begin{lstlisting}
M7: Vous avez gagne cette bataille, mais pas la guerre!
M8: Voulez-vous refaire une partie? (o-n)
					\end{lstlisting}
			\subsubsection{Mode graphique}
				\hspace{-3cm}
					\begin{tabular}{|p{1cm}|p{2.5cm}|p{1.5cm}|p{2.5cm}|p{2.5cm}|p{3cm}|p{3cm}|} %16cm
						\hline
						Label & Libellé & Type & Taille maximum & Est critique & Valeur par défault & Commentaire \\
						\hline \hline
						M7 & Message de fin & Text & une ligne & oui & Game Over & C'est le message de fin \\
						\hline
						M8 & Rejouer & Text & une ligne & oui & Voulez-vous refaire une partie? (o-n)& Permet à l'utilisateur de refaire une partie sans relancer le programme. \\
						\hline
					\end{tabular}
					\captionof{table}{Informations présentes sur l'écran d'accueil texte}
					\label{Informations présentes sur l'écran d'accueil texte}
					%---------------
					\subsubsection*{Maquette}
					\begin{figure}[!h] %on ouvre l'environnement figure
						\includegraphics[scale=1]{EF.png}
						\caption{Maquette de l'écran de fin graphique}
						\label{Maquette de l'écran de fin graphique}
					\end{figure}
					
	\section{Principales capacités}
		\subsection{Ecran d'accueil}
			\subsubsection{Mode texte}
				%tableau--------
				\hspace{-3cm}
				\begin{tabular}{|p{2.5cm}|p{2cm}|p{3cm}|p{2.5cm}|p{3cm}|p{3cm}|} %16cm
					\hline
					Action & Type & Résultat & Ecran de retour & Condition d'affichage & Règle de gestion \\
					\hline \hline
					Consultation des règles & Entrée standard & Affiche les règles AND poursuit le jeu & *lien vers les règles* & Lancement du programme &  Voir règle de gestion \ref{R1} et Erreur associé \ref{E1} \\
					\hline
				\end{tabular}
				\captionof{table}{Action possible sur l'écran d'accueil texte}
				\label{Action possible sur l'écran d'accueil texte}
				%---------------
			\subsubsection{Mode graphique}
				%tableau--------
				\hspace{-3cm}
				\begin{tabular}{|p{2.5cm}|p{2cm}|p{3cm}|p{2.5cm}|p{3cm}|p{3cm}|} %16cm
					\hline
					Action & Type & Résultat & Ecran de retour & Condition d'affichage & Règle de gestion \\
					\hline \hline
					Clic sur le bouton jouer & clic & Bascule sur l'écran de paramétrage & Affiche l'écran de paramétrage & Lancement du programme &  NA \\
					\hline
					Clic sur le bouton charger & clic & Ouvre l'explorateur de fichier & NA & Lancement du programme &  NA \\
					\hline
					Clic sur le bouton multijoueur & clic & Lance la recherche de joueurs potentiel & Ecran d'attente & Lancement du programme &  NA \\
					\hline
				\end{tabular}
				\captionof{table}{Action possible sur l'écran d'accueil graphique}
				\label{Action possible sur l'écran d'accueil graphique}
				%---------------
				
		\subsection{Ecran de paramétrage}
			\subsubsection{Mode texte}
				%tableau--------
				\hspace{-3cm}
				\begin{tabular}{p{|p{2.5cm}|p{2cm}|p{3cm}|p{2.5cm}|p{3cm}|p{3cm}|} %16cm
					\hline
					Action                        & Type                            				       & Résultat              	& Ecran de retour   		& Condition d'affichage 			& Règle de gestion \\
					\hline \hline
					Renter la hauteur du tableau & Entrée standard $$\mathbb{N} \in \left\{5;50\right\}$$  & Enregistre la hauteur 	& Question largeur  		& Arrive sur l'écran de paramétrage	& Voir règle de gestion	\ref{R2} et Erreur associé \ref{E2}\\
					\hline
					Rentrer la largeur du tableau & Entrée standard $$\mathbb{N} \in \left\{5;50\right\}$$  & Enregistre la largeur 	& Question obstacle 		& Répond à la question hauteur      & Voir règle de gestion \ref{R2} et Erreur associé \ref{E2}\\
					\hline
					Rentrer le nombre d'obstacles  & Entrée standard $$\mathbb{N} \in \left\{0;\frac{h*l}{2}\right\}$$  & Enregistre les obstacles  & Confirmation des paramètre& Répond à la question largeur & Voir règle de gestion \ref{R2} et Erreur associé \ref{E2}		\\
					\hline
					Confirmer vos choix 		  & Entrée standard (n'importe quelle touche)			   & Lance la partie		& Ecan de jeu				& Répond à la question obstacles	& Voir règle de gestion \ref{R1}	et Erreur associé \ref{E1}			\\
					\hline
				\end{tabular}
				\captionof{table}{Action possible sur l'écran de paramétrage texte}
				\label{Action possible sur l'écran de paramétrage texte}
				%---------------
			\subsubsection{Mode graphique}
				%tableau--------
				\hspace{-3cm}
				\begin{tabular}{|p{2.5cm}|p{2cm}|p{3cm}|p{2.5cm}|p{3cm}|p{3cm}|} %16cm
					\hline
					Action                        & Type                            				       & Résultat              	& Ecran de retour   		& Condition d'affichage 			& Règle de gestion \\
					\hline \hline
					Modifier la hauteur du tableau & Potentiomètre $$\mathbb{N} \in \left\{5;50\right\}$$  & Enregistre la hauteur 	& NA 		& Arrive sur l'écran de paramétrage	& NA\\
					\hline
					Modifier la largeur du tableau & Potentiomètre $$\mathbb{N} \in \left\{5;50\right\}$$  & Enregistre la largeur 	& NA 		& Arrive sur l'écran de paramétrage  & NA\\
					\hline
					Modifier le nombre d'obstacles & Potentiomètre $$\mathbb{N} \in \left\{0;\frac{h*l}{2}\right\}$$  & Enregistre les obstacles  & NA & Arrive sur l'écran de paramétrage & NA\\
					\hline
					Modifier le volume de la musique & Potentiomètre & Modifie le volume de la musique & NA & Arrive sur l'écran de paramétrage & NA\\
					\hline
					Modifier le volume des effets & Potentiomètre & Modifie le volume des effets & NA & Arrive sur l'écran de paramétrage & NA\\
					\hline
					Confirmer vos choix 		  & bouton		  & Lance la partie		& Ecan de jeu				& Arrive sur l'écran de paramétrage	& NA\\
					\hline
				\end{tabular}
				\captionof{table}{Action possible sur l'écran de paramétrage graphique}
				\label{Action possible sur l'écran de paramétrage graphique}
				%---------------

		\subsection{Ecran de jeu}
			\subsubsection{Mode texte}
				%tableau--------
					\hspace{-3cm}
					\begin{tabular}{|p{2.5cm}|p{2cm}|p{3cm}|p{2.5cm}|p{3cm}|p{3cm}|} %16cm
						\hline
						Action 					& Type 							& Résultat              		& Ecran de retour   		& Condition d'affichage & Règle de gestion \\
						\hline \hline
						Choix du robot			& Entrée standard (c OR p OR t) & Enregitre le robot choisi		& Choix de l'action 		& Début du tour de jeu 	& Voir règle de gestion \ref{R1} et Erreur associé\ref{E1}			\\
						\hline
						Choix de l'action 		& Entrée standard (t XOR d) 		& Enregistre l'action choisie 	& Choix de la direction 	& A choisi son robot	& Voir règle de gestion \ref{R1}  et Erreur associé \ref{E1}			\\
						\hline
						Choix de la direction 	& Entrée standard $$\mathbb{N} \in \left\{1;8\right\}$$ & Enregistre la direction choisie & Plateau de jeu	& A choisi l'action	&  Voir règle de gestion et Erreur associée \ref{E2} 		\\
						\hline
					\end{tabular}
					\captionof{table}{Action possible sur l'écran de jeu texte}
					\label{Action possible sur l'écran de jeu texte}
					%---------------
			\subsubsection{Mode graphique}
				%tableau--------
					\hspace{-3cm}
					\begin{tabular}{|p{2.5cm}|p{2cm}|p{3cm}|p{2.5cm}|p{3cm}|p{3cm}|} %16cm
						\hline
						Action 					& Type 							& Résultat              		& Ecran de retour   		& Condition d'affichage & Règle de gestion \\
						\hline \hline
						Choix du robot			& zone cliquable & Enregitre le robot choisi		& Affiche le type et l'énergie du robot restant & Début du tour de jeu 	& Voir règle de gestion \ref{R3} et Erreur associé\ref{E3}			\\
						\hline
						Choix de l'action 		& bouton & Enregistre l'action choisie 	& Le plateau avec les actions possibles mis en évidance & A choisi son robot	& Voir règle de gestion \ref{R4}  et Erreur associé \ref{E4}			\\ 
						\hline
						Tirer 					& zone cliquable & Soustrait la valeur de l'attaque du robot aux points de vie du robot adverse & Plateau mis à jour	& A choisi l'action	"Tirer" &  Voir règle de gestion \ref{R5} et Erreur associée \ref{E5}\\
						\hline
						Poser une mine 			& zone cliquable & Pose une mine aux coordonnées indiqué & Plateau mis à jour & A choisi l'action "poser une mine" & Voir règle de gestion \ref{R6} et Erreur associée \ref{E6} 		\\
						\hline
						Se déplacer 			& zone cliquable & Déplace le robot aux coordonnées indiqué & Plateau mis à jour & A choisi l'action "Mouvoir" & Voir règle de gestion \ref{R7} et Erreur associée \ref{E7} \\
						\hline
					\end{tabular}
					\captionof{table}{Action possible sur l'écran de jeu graphique}
					\label{Action possible sur l'écran de jeu graphique}
					%---------------

		\subsection{Ecran de fin de partie}
				\subsubsection{Mode texte}
					%tableau--------
					\hspace{-3cm}
					\begin{tabular}{|p{2.5cm}|p{2cm}|p{3cm}|p{2.5cm}|p{3cm}|p{3cm}|} %16cm
						\hline
						Action 			  & Type 					 &	Résultat 			 				 	  & Ecran de retour 		   & Condition d'affichage & Règle de gestion \\
						\hline \hline
						Choix recommencer & Entrée standard (o XOR n) & Recommence la partie ou ferme le programme & Ecran de parametrage  NA & Un des joueurs a perdu tout ces robots	& Voir règle de gestion \ref{R1} et Erreur associé \ref{E1} \\
						\hline
					\end{tabular}
					\captionof{table}{Action possible sur l'écran de fin de partie texte}
					\label{Action possible sur l'écran de fin de partie texte}
					%---------------
				\subsubsection{Mode graphique}
					%tableau--------
					\hspace{-3cm}
					\begin{tabular}{|p{2.5cm}|p{2cm}|p{3cm}|p{2.5cm}|p{3cm}|p{3cm}|} %16cm
						\hline
						Action 			  & Type 					 &	Résultat 			 				 	  & Ecran de retour 		   & Condition d'affichage & Règle de gestion \\
						\hline \hline
						Clic rejouer & bouton & Recommence la partie & Ecran de parametrage & Un des joueurs a perdu tout ces robots & NA \\
						\hline
						Clic Crédit & bouton & Affiche les crédits & Nom de tout ceux qui ont participé aux projet & Un des joueurs a perdu tout ces robots & NA \\
						\hline
					\end{tabular}
					\captionof{table}{Action possible sur l'écran de fin de partie graphique}
					\label{Action possible sur l'écran de fin de partie graphique}
					%---------------

	\section{Règles de Gestion}
		\begin{figure}[H]
			\caption{
			   	\label{R1} %Entrer un caractère
			   	L'utilisateur doit entrer un caractère valide: un seul caractère parmi ceux proposés.
			}
		\end{figure}

		\begin{figure}[H]
			\caption{
			  	\label{R2} %Enter un entier
			  	L'utilisateur doit entrer un entier valide: un seul entier compris dans l'intervalle indiqué dans la question.
			}
		\end{figure}

		\begin{figure}[H]
		   \caption{
		    	\label{R3} %Enter un entier
		    	Le robot doit avoir assez d'énergie pour effectuer l'action. Par exemple un piègeur avec 1 pts ne peut ni se déplacer ni poser une mine.
		    }
		\end{figure}

		\begin{figure}[H]
		   \caption{
		    	\label{R4} %Enter un entier
		    	Le robot doit pouvoir effectuer l'action. Par exemple un piègeur qui est entouré de 4 mines ne pourra pas poser une mine.
		    }
		\end{figure}

		\begin{figure}[H]
		   \caption{
		    	\label{R5} %Enter un entier
		    	Le robot doit pouvoir tirer sur sa cible. On ne peux pas tirer sur une case vide, un obstacle, un mur, un robot allié, une base ou soi-même. L'ennemie qui subit des dégâts est celui qui se trouve en premier sur la trajectoire du tir.
		    }
		\end{figure}

		\begin{figure}[H]
		   \caption{
		    	\label{R6} %Enter un entier
		    	Le piègeur doit pouvoir poser sur sa mine. On ne peut pas poser une mine sur un robot allié ou un ennemie, sur un obstacle, sur un mur, sur une base ou sur lui-même. On ne peux pas non plus poser de mine si le stock est vide.
		    }
		\end{figure}

		\begin{figure}[H]
		   \caption{
		    	\label{R7} %Enter un entier
		    	Le robot doit pouvoir se déplacer. Il ne peut pas: se diriger vers un mur, se diriger vers un obstacle se diriger vers la base adverse, se diriger vers sa base alors qu'il est le seul en dehors, se diriger vers un autre robot, se diriger vers sa propre case.
		    }
		\end{figure}

	\section{Messages d'erreurs}
	\begin{figure}[H]
		   \caption{
		    	\label{E1} %Entrer un caractère
		    	Affichage dans la sortie standard du message : Caractère incompatible! Veuillez entrer un caractère indiqué ou appuyer sur entrer. \\
		    	Code technique associé: Erreur T01.
			   }
		\end{figure}

		 \begin{figure}[H]
		   \caption{
		    	\label{E2} %Enter un entier
		    	Affichage dans la sortie standard du message : Valeur incompatible! Veuillez entrer un entier compris dans l'intervalle indiqué ou appuyer sur entrer. \\
		    	Code technique associé: Erreur T02.
		    }
		\end{figure}

		\begin{figure}[H]
		   \caption{
		    	\label{E3} %Enter un entier
		    	Affichage dans un pop-up : Sélection du robot impossible! \\
		    	Code technique associé: Erreur G01.
		    }
		\end{figure}

		\begin{figure}[H]
		   \caption{
		    	\label{E4} %Enter un entier
		    	Affichage dans un pop-up : Action impossible! \\
		    	Code technique associé: Erreur G02.
		    }
		\end{figure}

		\begin{figure}[H]
		   \caption{
		    	\label{E5} %Enter un entier
		    	Affichage dans un pop-up : Tirs impossible! \\
		    	Code technique associé: Erreur G03.
		    }
		\end{figure}

		\begin{figure}[H]
		   \caption{
		    	\label{E6} %Enter un entier
		    	Affichage dans un pop-up : Impossible de poser la mine! \\
		    	Code technique associé: Erreur G04.
		    }
		\end{figure}

		\begin{figure}[H]
		   \caption{
		    	\label{E7} %Enter un entier
		    	Affichage dans un pop-up :  Impossible de se déplacer dans cette direction\\
		    	Code technique associé: Erreur G05.
		    }
		\end{figure}

	\section{Scénario d'exploitation}
	\\
		%Organigramme d'execution----------
		\begin{tikzpicture}
			\usetikzlibrary{shapes, arrows}
			% style des nœuds
			\tikzstyle{debutfin}=[ellipse,draw,text=red] %debut/fin
			\tikzstyle{instruct}=[rectangle,draw,fill=green!50] %Intruction
			\tikzstyle{test}=[diamond, aspect=2.5,thick, draw=blue,fill=yellow!50,text=blue]
			\tikzstyle{es}=[rectangle,draw,rounded corners=4pt,fill=blue!25] %entrée/sortie
			% style des flèches
			\tikzstyle{suite}=[->,>=stealth’,thick,rounded corners=4pt]
			% placement des nœuds
			\node[debutfin] (debut) at (0,3) {Début}; %debut
			\node[instruct] (exe) at (0,2) {Lancement du programme};
			\node[es] (welcome) at (0,1) {Ecran d'accueil}; 
			\node[test] (testRead) at (0,-1) {Lire la documentation ?};
			\node[es] (read) at (-7,-1) {Ouverture de la documention};
			\node[instruct] (config) at (0,-4) {Paramétrage de la partie};
			\node[es] (print) at (0,-5) {Affichage du plateau};
			\node[instruct] (play) at (0,-6.25) {Tour de jeu (choix/déplacement/action)};
			\node[test] (endGame) at (0,-8) {Partie finie?};
			\node[es] (gameOver) at (0,-10.25) {Ecran de fin};
			\node[test] (replay) at (0,-12) {Rejouer?};
			\node[debutfin] (fin) at (0,-14) {Fin};
			% Placement des flèches 
			\draw[->] (debut) -- (exe);
			\draw[->] (exe) -- (welcome);
			\draw[->] (welcome) -- (testRead);
			\draw[suite] (testRead) -- (read) node[midway,fill=white]{oui};
			\draw[suite] (testRead) -- (config) node[midway,fill=white]{non};
			\draw[->] (config) -- (print);
			\draw[->] (print) -- (play);
			\draw[->] (play) -- (endGame);
			\draw[suite] (endGame) -- (-5,-8) -- (-5,-5) -- (print) node[midway,fill=white]{non};
			\draw[suite] (endGame)--(gameOver) node[midway,fill=white]{oui};
			\draw[->] (gameOver) -- (replay);
			\draw[suite] (replay) -- (fin) node[midway,fill=white]{non}; 
			\draw[suite] (replay) -- (-6,-12) -- (-6,-4) -- (config) node[midway,fill=white]{oui};
			\draw[->] (read) -- (-7,-14) -- (fin);
		\end{tikzpicture}
		%---------------------------------
	
	\section{Performance}
		Pour le jalon 1, nous nous ne péoccuperons pas des performances de l'application qui seront, dans le pire des cas, négligeables du fait de l'absence d'élément graphique. \\
		Pour la version finale, le jeu devra être fluide, et contenir peu ou pas de temps de chargement. L'IA ne doit pas mettre un temps raisonnablement court à jouer.
	\section{Contraintes imposées sur l'implémentation}
		Toutes les contraintes que nous nous sommes imposées sur implémentations sont fournies dans le document des normes techniques ci-joint.
	\section{Caractéristiques des utilisateurs}
		L'utilisateur peut être tout public à condition qu'il ait lu les règles du jeu au préalable. Aucune connaissance préalable est en informatique est requise.
	
\newpage 
\makeindex
\listofffigures
\listoftables
\end{document}
