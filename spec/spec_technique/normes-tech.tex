\documentclass{article}


\usepackage[utf8]{inputenc}
\usepackage[T1]{fontenc}
\usepackage[francais]{babel}
\usepackage{listings}
\usepackage{color}
\usepackage[top=3cm, bottom=2cm, left=5cm, right=4cm]{geometry}



\lstset{
language = Java,
numbers=left,
numbersep=7pt,
backgroundcolor=\color{white},
frame=single,
}



\title{Normes techniques}
\author{Aurélien \bsc{Svevi}}
\date{14 mars 2015}
\begin{document}
\maketitle

\tableofcontents
\newpage

\addcontentsline{toc}{part}{Normes de nommages}
\part*{Normes de nommages}
\section{Nommage des class et interfaces}

Les noms des class et interfaces doivent respecter les conventions suivantes:

\begin{enumerate}

\item Première lettre en majuscule
\item Première lettre de chaque mot en majuscule
\item Donner des noms simples et descriptifs
\item Éviter les acronymes
\item N'utiliser que les caractères [a-z] et [A-Z] et [0-9]

\end{enumerate}

\section{Nommage des variables}

Les noms des variables doivent respecter les conventions suivantes:

\begin{enumerate}

\item Première lettre en minuscule
\item Première lettre de chaque mot en majuscule
\item Donner des noms simples et descriptifs
\item N'utiliser que les caractères [a-z] et [A-Z] et [0-9]
\item Variables d'une seule lettre (pour un usage local):\begin{itemize}
	\item int : i, j, k, m, et n 
	\item char : c, d, et e 
	\item boolean : b
	\end{itemize}

\end{enumerate}

\section{Nommage des constantes}

Les noms des constantes doivent respecter les conventions suivantes:

\begin{enumerate}

\item Tout en majuscule
\item Séparer les mots par des underscore : '\_'
\item Donner des noms simples et descriptifs
\item N'utiliser que les lettres [A-Z], [0-9] et '\_'

\end{enumerate}

\section{Nommage des méthodes}

Les noms des méthodes doivent respecter les conventions suivantes: 

\begin{enumerate}
\item Le nom doit contenir un verbe
\item Première lettre en minuscule
\item Première lettre de chaque mot en majuscule
\item N'utiliser que les caractères [a-z] et [A-Z] et [0-9]
\end{enumerate}


\newpage
\addcontentsline{toc}{part}{Normes des fichiers}
\part*{Normes des fichiers}
\setcounter{section}{0}
\section{En-tête des fichiers}

Chaque fichier devra débuter par un commentaire contenant:

\begin{enumerate}
\item Le nom du développeur
\item La date de création du fichier
\item Une description rapide si nécessaire
\end{enumerate}




\section{Contenu des fichiers}


Un fichier ne devra pas contenir plus de 1500 lignes de codes.
\\
\\
Un fichier ne devra contenir qu'une seule class et le nom du fichier devra être de la forme "MaClasse.java".
\\
\\
Chaque ligne ne devra contenir qu'un seul traitement : Pas de ';' au milieu de la ligne sauf dans le cas des boucles for.

\begin{lstlisting}
res = 4 + 3;
i++;
\end{lstlisting}

\addcontentsline{toc}{part}{Normes de variables}
\part*{Normes des variables}

Chaque variable doit être définit sur une ligne
\begin{lstlisting}
String nom;
String prenom;
int age;
\end{lstlisting}

Toutes les variables utilisés dans un bloc doivent être définit au début de ce bloc (sauf les index de boucle)
\\
\\
Éviter de donner les mêmes noms à plusieurs variables définit dans des blocs différents (sauf variables locales).


\addcontentsline{toc}{part}{Normes de méthodes}
\part*{Normes des méthodes}

Une méthode ne devra pas prendre plus de 7 paramètres
\\
\\
Une méthode ne doit pas contenir plus de 150 lignes

\newpage
\addcontentsline{toc}{part}{Normes des instructions}
\part*{Normes des Instructions}
\setcounter{section}{0}
\section{L'instruction if}
L'instruction if devra se présenter sous cette forme:

\begin{lstlisting}
if (condition) {
	traitements;
} else if (condition) {
	traitements;
} else {
	traitements;
}
\end{lstlisting}

Elle devra obligatoirement contenir les accolades '\{' et '\}' même s'il n'y a qu'une seule ligne de traitement.


\section{L'instruction for}
L'instruction for devra se présenter sous cette forme:

\begin{lstlisting}
for (initialisation; condition; traitement) {
	traitements;
}
\end{lstlisting}

On peut également utiliser cette instruction sous la forme for each:

\begin{lstlisting}
for(Element e : List l){
	traitement;
}
\end{lstlisting}



\section{L'instruction switch}
L'instruction switch devra se présenter sous cette forme:

\begin{lstlisting}
switch (condition) {
case casA:
	traitements;
	break;
case casB:
	traitements;
	break;
case casC:
	traitements;
	break;
default:
	traitements;
	break;
}

\end{lstlisting}

\newpage
\addcontentsline{toc}{part}{Exemple de code}
\part*{Exemple de code}

\begin{lstlisting}
class MaSuperClass{

	private String nom;
	private int age;
	private final int ANNEE = 2015;


	public MaSuperClass(String nom, int age){
		this.nom = nom;
		this.prenom = prenom;
		this.age = age;
	}


	public int calculerAnneeNaissance(){
		return ANNEE-age;
	}


	public void afficherTrancheAge(){
		if(age < 18){
			System.out.println("Trop petit");
		} else if (age >= 18 && age <= 30) {
			System.out.println("Bienvenue");
		} else {
			System.out.println("Trop vieux");
		}

	}

}

\end{lstlisting}


\end{document}